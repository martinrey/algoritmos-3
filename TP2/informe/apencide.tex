\appendix

\lstset{language=C++,
	basicstyle=\ttfamily\footnotesize,
	keywordstyle=\color{blue}\ttfamily,
	stringstyle=\color{red}\ttfamily,
	commentstyle=\color{green}\ttfamily,
	numbers=left,
	breaklines=true,
	tabsize=1,
	commentstyle=\color{magenta},
	morecomment=[l][\color{magenta}]{\#}
}


\section{Código fuente: Problema 1} \label{App:AppendixA}

\begin{frame}

\begin{lstlisting}
size_t dijkstra(size_t origen, size_t destino, adyacencias_t& adyacencias, std::list<size_t>& itenerario) {

    std::priority_queue< Tupla > cola;

    std::vector<size_t> distancia(ciudades_.size(), INFINITO);
    std::vector<size_t> padre(ciudades_.size(), INFINITO);
    std::vector<size_t> vuelo(ciudades_.size(), INFINITO);

    distancia[origen] = 0;
    cola.push(Tupla(origen, 0));

    while (!cola.empty()) {
        Tupla u =cola.top(); cola.pop();
        for(auto& v: adyacencias[u.first]) {
            if (distancia[v.des_] > v.fin_) {
                if ((distancia[u.first] == 0) || ((int)distancia[u.first] <= (int)(v.ini_ - 2))) {
                    distancia[v.des_] = v.fin_;
                    padre[v.des_] = u.first;
                    vuelo[v.des_] = v.id_;
                    cola.push(Tupla(v.des_, distancia[v.des_]));
                }
            }
        }
    }

    if (distancia[destino] != INFINITO) {
        size_t ciudad = padre[destino];
        itenerario.push_front(vuelo[destino]);
        while (ciudad != origen && ciudad != INFINITO) {
            itenerario.push_front(vuelo[ciudad]);
            ciudad = padre[ciudad];
        }
        // Si la ultima ciudad no es el origen, entonces no hay itinerario.
        if (ciudad == INFINITO) {
            itenerario.clear();
            return INFINITO;
        }
    }
    return distancia[destino];
}

\end{lstlisting}
\end{frame}

\newpage
\section{Código fuente: Problema 2} \label{App:AppendixB}

\begin{frame}

\begin{lstlisting}
class Tablero {
	/// Casillas
	int n_;
	/// Cantidad de caballos
	int k_;
	/// Casilla (f, c) elegida como destino final
	int f_;
	int c_;
	/// Cantidad de movimientos totales necesarios para llevar a todos los caballos a la casilla (f, c)
	int m_;
    
	typedef pair<int, int> coord_t;
	typedef vector<vector<int> > matriz_t;
	vector<coord_t> caballos_;

	inline bool esValido(const coord_t& c, int fila, int col) const {
		return ((c.first+fila <= n_) && (c.first+fila > 0) && (c.second+col <= n_) && (c.second+col > 0));
	}

	inline coord_t sumar(const coord_t& c1, const coord_t& c2) const {
		return coord_t(c1.first + c2.first,  c1.second + c2.second);
	}

	void dameAdyacentes(const coord_t& c, list<coord_t>& adyacentes) const {
		// movimiento de caballo arriba derecha
		if (esValido(c, 1, 2))
			adyacentes.push_back(sumar(c, coord_t(1, 2)));
		// movimiento caballo arriba izquierda
		if (esValido(c, -1, 2))
			adyacentes.push_back(sumar(c, coord_t(-1, 2)));
		// movimiento de caballo derecha abajo
		if (esValido(c, 2, -1))
			adyacentes.push_back(sumar(c, coord_t(2, -1)));
		// movimiento caballo derecha arriba
		if (esValido(c, 2, 1))
			adyacentes.push_back(sumar(c, coord_t(2, 1)));
		// movimiento caballo abajo derecha
		if (esValido(c, 1, -2))
			adyacentes.push_back(sumar(c, coord_t(1, -2)));
		// movimiento caballo abajo izquierda
		if (esValido(c, -1, -2))
			adyacentes.push_back(sumar(c, coord_t(-1, -2)));
		// movimiento caballo izquierda arriba
		if (esValido(c, -2, 1))
			adyacentes.push_back(sumar(c, coord_t(-2, 1)));
		// movimiento caballo izquierda abajo
		if (esValido(c, -2, -1))
			adyacentes.push_back(sumar(c, coord_t(-2, -1)));
	}

	void bfs(coord_t origen, matriz_t &m){
		queue<coord_t> cola;

		vector<bool> aux(n_, false);
		vector<vector<bool > > visitados(n_, aux);
		cola.push(origen);
		visitados[origen.first-1][origen.second-1]  = true;
		
		while(!cola.empty()){
			coord_t cab = cola.front(); cola.pop();

			list<coord_t> adyacentes;
			dameAdyacentes(cab, adyacentes);

			for(auto &v: adyacentes){
				if (visitados[v.first-1][v.second-1] == false){
					visitados[v.first-1][v.second-1] = true;
					m[v.first-1][v.second-1] = m[cab.first-1][cab.second-1] + 1;

					cola.push(v);
				}
			}
		}
	}


public:
	Tablero(): n_(0), k_(0), f_(0), c_(0), m_(numeric_limits<int>::max()) {}

	void resolver(){
		// Para acumular la cantidad de movimientos
		matriz_t acumMovimientos(n_, vector<int>(n_, 0));
		// Para contar si cada caballo pas\'o por cada posici\'on (tienen que estar los k).
		matriz_t cantidad(n_, vector<int>(n_, 0));

		// Ejecuto k bfs (para cada caballo)
		for(int cab = 0; cab < k_; ++cab) {
			matriz_t movimientos(n_, vector<int>(n_, 0));

			bfs(caballos_[cab], movimientos);

			// Una vez que termino el BFS, acumulo los movimientos.
			for (int i = 0; i < n_; ++i) {
				for (int j = 0; j < n_; ++j) {
					acumMovimientos[i][j] += movimientos[i][j];
					// Ahora verifico que el caballo haya estado en esta posici\'on
					if (movimientos[i][j] > 0)
						++cantidad[i][j];
					else {
						if ( (caballos_[cab].first-1 == i) && (caballos_[cab].second-1 == j))
							++cantidad[i][j];
					}
				}
			}
		}

		// Busco la casilla que tenga la m\'inima cantidad de movimientos y todos los caballos lleguen a esa casilla.
		m_ = numeric_limits<int>::max();
		for (int i = 0; i < n_; ++i) {
			for (int j = 0; j < n_; ++j) {
				if (cantidad[i][j] == k_) {
					if (acumMovimientos[i][j] < m_) {
						m_ = acumMovimientos[i][j];
						f_ = i+1;
						c_ = j+1;
					}
				}
			}
		}

	}

	void clear() {
		 f_ = 0;
		 c_ = 0;
		 m_ = numeric_limits<int>::max();
	}

	int size_n() const {
		return n_;
	}

	int size_k() const {
		return k_;
	}

};
\end{lstlisting}
\end{frame}

\newpage
\section{Código fuente: Problema 3} \label{App:AppendixC}

\begin{frame}

\begin{lstlisting}
class Nodo{
public:
    int numero, distancia;
    bool visitado, en_anillo;
    Nodo * padre;
    vector <pair <Nodo*, int> > adyacentes;
    vector <int> arista_en_AGM;

    Nodo(int numero, int N): numero(numero){
        padre = NULL;
        distancia = numeric_limits<int>::max();
        visitado = false;
        en_anillo = false;
        arista_en_AGM.resize(N+1);
    }

    void agregarAdyacente(Nodo* adyacente, int costo) {
        adyacentes.push_back(make_pair(adyacente, costo));
    }
};


struct Comp {
    bool operator()(const Nodo* s1, const Nodo* s2) {
        return (s1->distancia > s2->distancia);
    }
};

class LCdA{
public:
	int N, M, C;
	vector<Nodo> nodos;
    vector<pair < int , pair <Nodo*, Nodo* > > > aristas ;
    vector<pair<Nodo*, Nodo*> > anillo;
    vector<pair<Nodo*, Nodo*> > resto;

	LCdA():C(0){}
	int buscar_AGM() {
    
        // Si M < N no puede ser conexo y tener un ciclo
        if(M < N)
            return 1;

        vector <Nodo*>  heap;
        for (int i = 1; i <= N; i++)
            heap.push_back(&nodos[i]);

        Nodo * u = heap[0];
        u->distancia = 0;

        while (!heap.empty()) {   
            Nodo * u = heap.front(); pop_heap(heap.begin(), heap.end(), Comp()); 
            heap.pop_back();

            // Distancia es infinito <==> es no conexo
            if(u->distancia == numeric_limits<int>::max())
                return 1;

            // Anoto la arista como parte del AGM
            if (u->padre != NULL)        {
                u->arista_en_AGM[u->padre->numero] = 1;
                u->padre->arista_en_AGM[u->numero] = 1;
            }

            C+= u->distancia;
            u->visitado = true;

            for (unsigned int i = 0; i < u->adyacentes.size(); i++) {
                Nodo * v = u->adyacentes[i].first;
                if (!v->visitado && v->distancia > u->adyacentes[i].second) {
                    v->padre = u;
                    v->distancia = u->adyacentes[i].second;
                }
            }
            make_heap(heap.begin(), heap.end(), Comp());
        }
        return 0;
    }

	void formar_anillo() {
        sort(aristas.begin(), aristas.end());
        pair<Nodo*, Nodo*> arista_extra;

        for (unsigned int i = 0; i < aristas.size(); i++) {
            if(aristas[i].second.first->arista_en_AGM[aristas[i].second.second->numero] == 0) {
                C += aristas[i].first;
                arista_extra = aristas[i].second;
                break;
            }
        }

        Nodo * a = arista_extra.first;
        Nodo * b = arista_extra.second;

        // Formo el anillo llendo desde cada uno de los nodos de la arista extra hacia atras hasta llegar al otro nodo o a NULL
        for (int i = 0; i < 2; i++) {        
            while(a != NULL && a != b) {
                a->en_anillo = true;
                a = a->padre;
            }
            a = arista_extra.second;
            b = arista_extra.first;
        }

        anillo.push_back(arista_extra);

        // Separo las aristas del resto de las del anillo
        for (int i = 1; i <= N; i++) {
            Nodo* k = &nodos[i];
            if (k->padre != NULL) {
                if (k->en_anillo)
                    anillo.push_back(make_pair(k, k->padre));
                else
                    resto.push_back(make_pair(k, k->padre));
            }
        }
    }

};

\end{lstlisting}
\end{frame}

