\appendix

\lstset{language=C++,
	basicstyle=\ttfamily\footnotesize,
	keywordstyle=\color{blue}\ttfamily,
	stringstyle=\color{red}\ttfamily,
	commentstyle=\color{green}\ttfamily,
	numbers=left,
	breaklines=true,
	tabsize=1,
	commentstyle=\color{magenta},
	morecomment=[l][\color{magenta}]{\#}
}


\section{Ejercicio 1} \label{App:AppendixA}

\begin{frame}

\begin{lstlisting}
void puentes(const std::vector<bool>& tablones, uint16_t c, std::vector<uint16_t>& saltos) {
	uint16_t j = 0, k = 0, n = tablones.size();

	// caso que puedo saltar todo de una
	if (c < n){
		// no puedo saltar todo de una
		while (j < n){
			for(k = c; k > 0; k--){
				if ((j+k-1 < n) && (tablones[j+k-1] != false)){
					saltos.push_back(j+k);
					break;
				}
			}
			if (k == 0){ // "No hay solucion";
				saltos.clear();
				break;
			}

			j += k;
		}
	}
	else {
		saltos.push_back(n);
	}
}

\end{lstlisting}
\end{frame}

\newpage
\section{Ejercicio 2} \label{App:AppendixB}

\begin{frame}

\begin{lstlisting}
typedef struct edificio_st {
	size_t izq;
	size_t alt;
	size_t der;
	edificio_st() : izq(0), alt(0), der(0) {}
	edificio_st(size_t i, size_t a, size_t d) : izq(i), alt(a), der(d) {}
} edificio_t;

// Comparacion para el heap.
struct Comp {
	bool operator()(const edificio_t& s1, const edificio_t& s2) {
		return s1.alt <= s2.alt;
	}
};

// de menor a mayor con los valores de la izq
bool orden(const edificio_t& a, const edificio_t& b) {
	return ((a.izq < b.izq) ||
			(a.izq == b.izq && a.alt > b.alt) ||
			(a.izq == b.izq && a.alt == b.alt && a.der < b.der));
}

// de mayor a menor con los valores de la der
bool orden2(const edificio_t& a, const edificio_t& b) {
	return ((a.der > b.der) ||
			(a.der == b.der && a.alt > b.alt) ||
			(a.der == b.der && a.alt == b.alt && a.izq > b.izq));
}

class Horizonte {
public:
	Horizonte() : n(0) {
	}

	void resolver() {
		flancosAscendentes(piso);
		flancosDescendentes(piso);
		sort(solucion.begin(), solucion.end());
		// Para borrar los repetidos
		solucion.erase(unique(solucion.begin(), solucion.end()), solucion.end());    
	}

	void clear() {
		solucion.clear();
	}

	size_t size() const {
		return n;
	}

	friend ostream & operator<<(ostream &os, Horizonte& h);
	friend istream & operator>>(istream &is, Horizonte& h);

protected:
	size_t n;
	vector<edificio_t> edificios;
	vector<pair<size_t, size_t> > solucion;
	edificio_t piso;

	void flancosAscendentes(const edificio_t& piso) {
		vector<edificio_t> heap;

		make_heap(heap.begin(), heap.end(), Comp()); // Notar que al comienzo heap esta vacio.
		heap.push_back(piso);
		push_heap(heap.begin(), heap.end(), Comp());

		sort(edificios.begin(), edificios.end(), orden);

		for (size_t i = 0; i < n; i++) {
			edificio_t edi = heap.front();
			while (edificios[i].izq >= edi.der) {
				pop_heap(heap.begin(), heap.end(), Comp());
				heap.pop_back();
				edi = heap.front();
			}

			if (edi.alt < edificios[i].alt)
				solucion.push_back(make_pair(edificios[i].izq, edificios[i].alt));

			heap.push_back(edificios[i]);
			push_heap(heap.begin(), heap.end(), Comp());
		}
	}

	void flancosDescendentes(const edificio_t& piso) {
		vector<edificio_t> heap;

		make_heap(heap.begin(), heap.end(), Comp());
		heap.push_back(piso);
		push_heap(heap.begin(), heap.end(), Comp());

		sort(edificios.begin(), edificios.end(), orden2);

		for (size_t i = 0; i < n; i++) {
			edificio_t edi = heap.front();
			while (edificios[i].der < edi.izq) {
				pop_heap(heap.begin(), heap.end(), Comp());
				heap.pop_back();
				edi = heap.front();
			}

			if (edi.alt < edificios[i].alt)
				solucion.push_back(make_pair(edificios[i].der, edi.alt));

			heap.push_back(edificios[i]);
			push_heap(heap.begin(), heap.end(), Comp());
		}
	}
};
\end{lstlisting}
\end{frame}

\newpage
\section{Ejercicio 3} \label{App:AppendixC}

\begin{frame}

\begin{lstlisting}
typedef struct sustancia_st {
	uint16_t numero;
	vector<uint16_t> coeficiente;
	sustancia_st(uint16_t i, uint16_t n) : numero(i)  {
		coeficiente.resize(n);
	}
} sustancia_t;

class Camion {
public:
	vector<sustancia_t*> sustanciasTransportadas; // Sustancias transportadas por el camion
	uint16_t peligrosidad; // Nivel de peligrosidad de las sustancias transportadas

	/// Constructor
	Camion() : peligrosidad(0) {
	}

	/// Agrega sust al camion y le sube la peligrosidad
	void agregarSustancia(sustancia_t * sust) {
		peligrosidad += calcularPeligrosidad(sust);
		sustanciasTransportadas.push_back(sust);
	}

	/// Saca "sust" del camion y baja la peligrosidad. Siempre es la ultima sustancia agregada por el contexto en que la uso.
	void sacarSustancia(sustancia_t * sust)  {
		sustanciasTransportadas.pop_back();
		peligrosidad -= calcularPeligrosidad(sust);
	}

	/// Chequea si al agregar "sust" al camion no se pasa el umbral de peligrosidad
	bool noEsPeligroso(const uint16_t &umbral, const sustancia_t * sust) {
		uint16_t x = calcularPeligrosidad(sust) + peligrosidad;
		return (x <= umbral);
	}

private:
	/// Calcula la peligrosidad de agregar "sust" al camion (no la agrega, solo la calcula)
	uint16_t calcularPeligrosidad(const sustancia_t * sust) {
		uint16_t suma = 0;
		for (auto& st: sustanciasTransportadas)
			suma += sust->coeficiente[st->numero];
		return suma;
	}
};

class Biohazard {
public:
	Biohazard() :
		n(0), m(0),
		minimosCamionesUsados(numeric_limits<uint16_t>::max()),
		usadas(0) {
	}

	void backtracking() {
		camiones.resize(n);
		recursion(0);
		return;
	}

	void clear() {
		camiones.clear();
		solucion.clear();
		transportados.clear();
		usadas = 0;
	}

	uint16_t size() const {
		return n;
	}

	friend ostream & operator<<(ostream &os, Biohazard& b);
	friend istream & operator>>(istream &is, Biohazard& b);

protected:
	uint16_t n;
	uint16_t m;
	uint16_t minimosCamionesUsados;

	vector<sustancia_t> productos;
	vector<Camion> camiones;
	vector<uint16_t> transportados;
	uint16_t usadas; /// Cantidad de sustancias utilizadas
	vector<uint16_t> solucion;

	void recursion(const uint16_t &x) {
		uint16_t contador = contar();

		if (contador < minimosCamionesUsados) {

			if (todosUsados()) {
				minimosCamionesUsados = contador;
				solucion = transportados;
			}

			for (uint16_t i = x; i < n; i++) { // i itera las sustancias
				if (transportados[i] == 0)   { // Si no esta en ningun camion...
					sustancia_t * sust = &productos[i];
					for (uint16_t c = 0; c < n; c++) { // c itera los camiones
						Camion * truck = &camiones[c];
						 // Si no supera el umbral de peligrosidad al agregar la sustancia sust al truck...
						if (truck->noEsPeligroso(m, sust)) 
							// Agrego sust al camion{
							truck->agregarSustancia(sust); 
							usar(sust->numero, c + 1); // Agendo sust como ya transportada
							recursion(i);
							// Vuelvo a dejar como si no lo lo hubiese agregado, para probar con la siguiente sustancia
							desUsar(sust->numero);
							truck->sacarSustancia(sust);
						}
					}
				}
			}
		}

	}

	/// Agenda "sust" como ya transportada por el camion "c"
	void usar(uint16_t sust, uint16_t c) {
		transportados[sust] = c;
		++usadas;
	}

	/// Agenda "sust" como no transportada por ningun camion
	void desUsar(uint16_t sust) {
		transportados[sust] = 0;
		--usadas;
	}

	/// Devuelve true <==> estan todas las sustancias en algun camion
	inline bool todosUsados() {
		return (usadas == n);
	}

	/// Para saber cuantos camiones se usaron y tambien sirve para un par de "podas"
	uint16_t contar() {
		uint16_t res = 0;
		for (uint16_t i = 0; i < n; i++)
			res = max(res, transportados[i]);
		return res;
	}
};

\end{lstlisting}
\end{frame}

